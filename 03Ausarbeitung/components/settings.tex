% Included by MAIN.TEX
% Defines the settings for the CAMP report document

\renewcommand{\sectfont}{\normalfont \bfseries}        % Schriftart der Kopfzeile

% manipulate footer
\usepackage{scrpage2}
\pagestyle{scrheadings}
\ifoot[\footertext]{\footertext} % \footertext set in INFO.TEX
%\setkomafont{pagehead}{\normalfont\rmfamily}
\setkomafont{pagenumber}{\normalfont\rmfamily}

%% allow sophisticated control structures
\usepackage{ifthen}

% use Palatino as default font
\usepackage{palatino}

% enable special PostScript fonts
\usepackage{pifont}

% make thumbnails
\usepackage{thumbpdf}

%to use the subfigures
\usepackage{subfigure}


\usepackage{colortbl}

%for code snippets
\usepackage{listings}
\definecolor{mygreen}{rgb}{0,0.6,0}
\definecolor{mygray}{rgb}{0.5,0.5,0.5}
\definecolor{mymauve}{rgb}{0.58,0,0.82}
\lstset{ %
  backgroundcolor=\color{white},   % choose the background color
  basicstyle=\footnotesize\ttfamily,        % size of fonts used for the code
  breaklines=true,                 % automatic line breaking only at whitespace
  captionpos=b,                    % sets the caption-position to bottom
  commentstyle=\color{mygreen},    % comment style
  escapeinside={\%*}{*)},          % if you want to add LaTeX within your code
  keywordstyle=\color{blue},       % keyword style
  stringstyle=\color{mymauve},     % string literal style
}

%for outer join
\usepackage{ifsym}
%for tree
\usepackage{qtree}

%% show program code\ldots
%\usepackage{verbatim}
%\usepackage{program}

%% enable TUM symbols on title page
\usepackage{styles/tumlogo}


\usepackage{multirow}

%% use colors
\usepackage{color}

%% make fancy math
\usepackage{amsmath}
\usepackage{amsfonts}
\usepackage{amssymb}
\usepackage{textcomp}
\usepackage{yhmath} % f�r die adots 
%% mark text as preliminary
%\usepackage[draft,german,scrtime]{prelim2e}

%% create an index
\usepackage{makeidx}

% for the program environment
\usepackage{float}

%% load german babel package for german abstract
%\usepackage[german,american]{babel}
\usepackage[german,english]{babel}
\selectlanguage{german}

% use german characters as well
%\usepackage[latin1]{inputenc}       % allow Latin1 characters
\usepackage[T1]{fontenc}
\usepackage[utf8]{inputenc}

 
% use initals dropped caps - doesn't work with PDF
\usepackage{dropping}


\usepackage{styles/shortoverview}
%----------------------------------------------------
%      Graphics and Hyperlinks
%----------------------------------------------------

%% check for pdfTeX
\ifx\pdftexversion\undefined
 %% use PostScript graphics
 \usepackage[dvips]{graphicx}
 \DeclareGraphicsExtensions{.eps,.epsi}
 \graphicspath{{figures/}{figures/review}} 
 %% allow rotations
 \usepackage{rotating}
 %% mark pages as draft copies
 %\usepackage[english,all,light]{draftcopy}
 %% use hypertex version of hyperref
 \usepackage[hypertex,hyperindex=false,colorlinks=false]{hyperref}
\else %% reduce output size \pdfcompresslevel=9
 %% declare pdfinfo
 %\pdfinfo { 
 %  /Title (my title) 
 %  /Creator (pdfLaTeX) 
 %  /Author (my name) 
 %  /Subject (my subject	) 
 %  /Keywords (my keywords)
 %}
 %% use pdf or jpg graphics
 \usepackage[pdftex]{graphicx}
 \DeclareGraphicsExtensions{.jpg,.JPG,.png,.pdf,.eps}
 \graphicspath{{figures/}} 
 
 %% Load float package, for enabling floating extensions
 \usepackage{float}
 
 %% allow rotations
 \usepackage{rotating}
 %% use pdftex version of hyperref
% \usepackage[pdftex,colorlinks=true,linkcolor=red,citecolor=red,%
% anchorcolor=red,urlcolor=red,bookmarks=true,%
% bookmarksopen=true,bookmarksopenlevel=0,plainpages=false%
% bookmarksnumbered=true,hyperindex=false,pdfstartview=%
% ]{hyperref}
%
\usepackage[pdftex,colorlinks=false,linkcolor=red,citecolor=red,%
 anchorcolor=red,urlcolor=red,bookmarks=true,%
 bookmarksopen=true,bookmarksopenlevel=0,plainpages=false%
 bookmarksnumbered=true,hyperindex=false,pdfstartview=%
 ]{hyperref}
\fi

%for pgfplot
%\input pgfplots.tex % Plain TeX
\usepackage{pgfplots} % LaTeX
%\usemodule[pgfplots] % ConTeXt



%% Fancy chapters
%\usepackage[Lenny]{fncychap}
%\usepackage[Glenn]{fncychap}
%\usepackage[Bjarne]{fncychap}

%\usepackage[avantgarde]{quotchap}

% set the bibliography style
%\bibliographystyle{styles/bauermaNum}
%\bibliographystyle{alpha}
\bibliographystyle{plain}
